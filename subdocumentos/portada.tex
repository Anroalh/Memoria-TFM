%----------
%	PORTADA
%----------	
\begin{titlepage}
	\begin{sffamily}
		\color{azulUC3M}
		\begin{center}
			\begin{figure}[H] %incluimos el logotipo de la Universidad
				\makebox[\textwidth][c]{\includegraphics[width=16cm]{Portada_Logo.png}}
			\end{figure}
			\vspace{1cm}
			\begin{Large}
				Máster en Ingeniería Industrial\\			
				2021-2022\\
				\vspace{1cm}		
				\textsl{Trabajo Fin de Máster}
				\bigskip
				
			\end{Large}
			{\Huge Detección y reconocimiento de imagen con aplicación ferroviaria para la validación de equipos industriales y operación automática}\\
			\vspace*{0.5cm}
			\rule{10.5cm}{0.1mm}\\
			\vspace*{0.9cm}
			{\LARGE Antonio Rodríguez Alhambra}\\ 
			\vspace*{1cm}
			\begin{Large}
				Tutor/es\\
				M. Isabel Herreros Cid\\
				Miguel López Hernández\\
				Lugar y fecha de presentación prevista\\
			\end{Large}
		\end{center}
		
		\vfill
		\color{black}
		\begin{footnotesize}
			\noindent\fbox{
				\begin{minipage}{\textwidth}
					\textbf{DETECCIÓN DEL PLAGIO}\\
					La Universidad utiliza el programa \textbf{Turnitin Feedback Studio} para comparar la originalidad del trabajo entregado por cada estudiante con millones de recursos electrónicos y detecta aquellas partes del texto copiadas y pegadas. Copiar o plagiar en un TFM es considerado una \textbf{\underline{Falta Grave}}, y puede conllevar la expulsión definitiva de la Universidad.
				\end{minipage}	
			}
			\vspace*{.5cm}\\	
			\noindent\includegraphics[width=4.2cm]{imagenes/creativecommons.png}\\
			\emph{[Incluir en el caso del interés en su publicación en el archivo abierto]}\\
			Esta obra se encuentra sujeta a la licencia Creative Commons \textbf{Reconocimiento - No Comercial - Sin Obra Derivada}
			
		\end{footnotesize}
	\end{sffamily}
\end{titlepage}

\newpage %página en blanco o de cortesía
\thispagestyle{empty}
\mbox{}