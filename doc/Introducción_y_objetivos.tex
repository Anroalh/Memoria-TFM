\chapter{Introducción}

QUEDA POR ESCRIBIR:
\begin{itemize}
	\item Resumen (lo último)
	\item Introducción: 
	\begin{itemize}
		\item Motivación del trabajo
		\begin{itemize}
			\item Hablar del problema del HMI, de porqué se quiere implementar un sistema que reconozca los símbolos en tiempo real de la pantalla por áreas
			\item Qué áreas son las que componen el HMI.
		\end{itemize}
	\end{itemize}
	\item Objetivos: aunque estén escritos, poner muchos más de los que hablar en las conclusiones sobre si se han cumplido o no.
	\item Estado del arte
	\begin{itemize}
		\item Está hecha una introducción a la inteligencia artificial, distinguiendo entre machine learning y deep learning (quizás añadir alguna imagen).
		\item Está hecho el porqué de elegir Python, pero queda poner porqué se ha usado Machine Learning y porqué Scikit-learn (EN DETALLE)
		\item Hablar de la aplicación de machine learning en el ámbito ferroviario, porqué es importante y donde se aplica (pruebas de laboratorio, pruebas de ensayo, etc) (aquí necesito la ayuda de Miguel para que me informe)
	\end{itemize}
	\item Discusión de los resultados: necesito el visto bueno de Miguel sobre la primera versión de los resultados. Si no me lo puede dar hago una versión inicial de este apartado.
	\item Impacto socio-económico: calcular (inventado, a ver quién sabe los números exactos) el coste del proyecto.
	\item Impacto social: importancia del transporte ferroviario, importancia de las pruebas de ensayo e importancia de automatizar en el sector ferroviario (necesitar una pequeña guía de miguel sobre a donde enfocar esto)
	\item Conclusiones finales (lo penúltimo antes del resumen)
	\item Futuras líneas del trabajo (pedir ideas a Miguel)
	\begin{itemize}
		\item Mejoras del trabajo: principalmente hablar de obtener mejores imágenes, probar todos los posibles modelos de aprendizaje supervisado una vez se tenga un dataset fiel al que se vaya a utilizar de verdad.
		\item Ampliación del estudio: lo que pone encima de probar con más modelos, decir cuales podrían ir (porqué) y cuáles no (porqué no) (los árboles de decisiones que sobre ajustan que da gusto).
		\item Problemas sin resolver: que yo sepa que distinga el color, el medidor de velocidad y las áreas que no sean símbolos sin más.
	\end{itemize}
\end{itemize}


\chapter{Objetivos}

El principal objetivo de este Trabajo de Fin de Máster es encontrar un algoritmo de clasificación multiclase con un alto rendimiento de discriminación entre las clases de cada área del HMI.

\begin{description}
	\item[Métodos de extracción de características] Investigar e implementar métodos de obtención de datos, sobre las imágenes a discriminar, que obtengan información que permita al clasificador separar entre clases.
	\item[Sencillez en la ejecución] Desarrollar un proceso de preparado de datos, selección y clasificación que no requiera de procesos extensivos y computacionalmente pesados para lograr el mejor resultado con la mayor simplicidad posible. Además, que no requiera de grandes conocimientos en el área de \textit{machine learning}, de tal forma que el proceso pueda ser modificado o expandido en un futuro sin requerir formación en el campo.
	\item[Aprendizaje del campo] Aún cuando la formación del autor es en ingeniería industrial, electrónica y automática, y el área de \textit{machine learning} queda fuera de las enseñanzas de esta disciplina, un objetivo de este trabajo es una puesta en contacto con un área de investigación y desarrollo cada vez más en auge en el mundo actual, debido a su increíble potencial en casi cualquier área que requiera de clasificación o regresión de datos.
	\item[Realización del Trabajo para finalización del curso] Completar satisfactoriamente el proyecto para su presentación en el año de graduación y terminación del Máster.
\end{description}

\mynote{AÚN por terminar.....}